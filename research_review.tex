\documentclass[12pt, a4paper]{article}
\usepackage[a4paper, margin=2cm]{geometry}
\usepackage{tabularx}
\usepackage{booktabs}
\usepackage{graphicx}
\usepackage{minted}
\usepackage[backend=bibtex]{biblatex}
\setminted{fontsize=\small,baselinestretch=1}
\bibliography{./bibliography}{}
\author{Andrzej Wytyczak-Partyka}
\title{AIND Planning research review}
\begin{document}
\maketitle

\section{Introduction}
Planning has been in the eye of AI community since the very beginning of AI.
Planning applications have been driving research in AI for years. Below is a brief description
of 3 important developments in the field of AI planning.

\section{Graphplan}
The birth of Graphplan algorithm in 1995 \cite{blum1997fast} has moved the research around planning away from
partial-order planning and allowed the researchers to broaden the area of applications by translating
various problems to planning problems and using solvers to solve them.

\section{PDDL}
The development and wide adoption of Planning Domain Definiton Language \cite{mcdermott1998pddl} has enabled researchers
to benchmark their algorithms more easily. Before PDDL - languages like STRIPS or ADL and their multiple
extensions and flavors have been used but the multitude of languages made comparing algorithms
more difficult.

\section{HSP}
In 1998 the Heuristic Search Planner \cite{bonet1999planning} has shifted the research from backward search
to forward search (progression planners), which until 1998 seemed unfeasible
because of large computational effort (e.g. large branching factor). HSP has shown
how relaxing the original planning problem can help create an admissible heuristic
which can then be used e.g. for A*.


%\begin{thebibliography}
  %\bibliographystyle{unsrt}
%\end{thebibliography}
\printbibliography

\end{document}
